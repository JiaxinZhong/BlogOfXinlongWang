\documentclass[UTF8]{ctexbook}

% SCI book 的模板 sty,适用于中文
\usepackage{scibookchn}
\usepackage{scigeneral}


\begin{document}

\chapter{声波的基本性质}

\section{全反射状态的声场与声能流}
英文标题:
Sound Fields and Energy Flux Under Total Reflection

Web:
http://xlwangnu.blog.163.com/blog/static/190719270201192502043374/

\emph{摘要}
全反射是常见声光现象。在媒质交界面上,若入射媒质的声速不如透射媒质的(或折
射率小于1),则当入射角足够大就会发生全反射,入射声能悉数反射。虽然如此,
全反射下声场如何分布?透射媒质中是否存在声场?如存在,取何种波动方式?此
正本文试图回答的问题。分析表明,在全反射下,(1)入射媒质中的总声场在界面法
向呈驻波形式,而在界面切向呈行波形式;(2)仍存在透射声波,但以沿界面传播的
表面波形式存在,但垂直于界面的方向上指数衰减。

若透射媒质的声速$c_2$大于入射媒质的声速$c_1$,且入射角(incident angle)
$\theta\st{i}$大于全反射临界角$\theta\st{ic}$,
\begin{equation}
\theta\st{ic} = \arcsin \frac{c_1}{c_2} = \arcsin n, \qty(n=\frac{c_1}{c_2}
<1)
\end{equation}
声波全部反射。此即著名的\emph{声全反射现象}(total reflection)。与光
的全反射略有不同,声的全反射一般发生在“硬界面”,如从空气到水,从水到固体。
光的全反射则发生在“软表面”---从光密介质入射到光疏介质,如光纤内的光反射,故
而一般称之为全内反射(Total internal reflection)。或问:在全反射状态,声场
分布如何?声能如何传播?此正本文所欲讨论的。

\subsection{声场分布}
如图xx所示,频率为$\omega$的平面声波从声特性阻抗率为$\rho_1c_1$的媒质中以角
度$\theta\st{i}$入射到声特性阻抗率为$\rho_2c_2$的媒质上。
设入射方向和两媒质边界面的法向构成的平面为$(x,y)$平面,$x$轴沿边界内法向,
$y$轴在边界平面上。根据平面行波论,入射(incident)、反射(reflected)和
透射(transmitted)的声场解各具如下形式:
\begin{equation}
	\label{eq:totalref_pressure}
	\begin{cases}
	p\st{i} = p\st{ia} \me^{-\mj k_1 \vb{n}\st{i} \vdot \vb{r}}
	\qc \vb{v}\st{i} = \dfrac{p\st{i}}{\rho_1c_1} \vb{n}\st{i}\\
	p\st{r} = p\st{ra} \me^{-\mj k_1 \vb{n}\st{r} \vdot \vb{r}}
	\qc \vb{v}\st{r} = \dfrac{p\st{r}}{\rho_1c_1} \vb{n}\st{r}
	\qc (p\st{ra} = r_p p\st{ia}) \\
	p\st{t} = p\st{ta} \me^{-\mj k_2 \vb{n}\st{t} \vdot \vb{t}}
	\qc \vb{v}\st{t} = \dfrac{p\st{t}}{\rho_2c_2} \vb{n}\st{t}
	\qc (p\st{ta} = t_p p\st{ta}) 
	\end{cases}
\end{equation}
式中,$p\st{ia}$、$p\st{ra}$和$p\st{ta}$分别为入射、反射和透射(折射)
波的声压幅度,$r_p$和$t_p$分别为声压反射系数和透射系数,$\theta\st{t}$为
折射角(refracted angle),$k_1$和$k_2$分别为入射和透射媒质的波数,
而$\vb{n}\st{i}$、$\vb{n}\st{r}$和$\vb{n}\st{t}$则依次为入射、反射和
折射波的方向矢量,
\begin{equation}
	\vb{n}\st{i}=(\cos \theta\st{i}, \sin \theta\st{i})\qc 
	\vb{n}\st{r}=(-\cos \theta\st{r}, \sin \theta\st{r})\qc 
	\vb{n}\st{t}=(\cos \theta\st{t}, \sin \theta\st{t})
\end{equation}
在$\vb{n}\st{r}$表达式中,已利用了关系$\vb{theta}\st{r}=\theta\st{i}$。为
简洁起见,本文一根省略时间因子$\exp(\mj \omega t)$。

根据Snell定律(折射律),
\begin{equation}
	\label{eq:totalref_snell}
	\sin\theta\st{t} = \frac{\sin\theta\st{i}}{n}\qc (n<1)
\end{equation}
当$\theta>\theta\st{ic}$时,$\sin\theta\st{t}>1$,$\theta\st{t}$不复为实数
而是复数。是以,作复变换:$\theta\st{t} \rightarrow \psi$,
\begin{equation}
	\label{eq:totalref_theta_trans}
	\theta\st{t} = \frac{\uppi}{2}+\mj \psi\qc 
	\begin{cases}
		\sin \theta\st{t} = \cosh \psi\\
		\cos\theta\st{t} = -\mj \sinh \psi
	\end{cases}
\end{equation}
代入折射公式(\ref{eq:totalref_snell}),则 Snell定律改为
\begin{equation}
	\cosh \psi = \frac{\sin \theta\st{i}}{n}
\end{equation}
由此可知,$\psi$具有实数解。当$\theta\st{i}$从$\theta\st{ic}$增至$\uppi/2$
时,$\psi$从$0$单调增大至$\arccosh(1/n)>0$。此处杂提醒读者,数学上,变换
(\ref{eq:totalref_theta_trans})中的$\psi$可正可负。但考虑到下面公式xx(12)
给出的透射声压解$p\st{t}$中,声波必沿$x$正向衰减,$\psi$只能取正值。引入变
换(\ref{eq:totalref_theta_trans})后,全反射($\theta\st{i}>\theta\st{ic}$)
状态下的声压反射系数$r_p$和透射系数$t_p$可表为\footnote{编者著:利用$x=0$
处的边界条件,即$\begin{cases}1+r_p = t_p\\ \dfrac{1-r_p}{z\st{s1}} =
	\dfrac{t_p}{z\st{s2}}\end{cases}$,其中法向声阻抗率分别是
$z\st{s1}= \rho_1c_1/\cos \theta\st{i}, z\st{s2} = \rho_2c_2
/\cos\theta\st{i}$,得到$r_p = \dfrac{z\st{s2}-z\st{s1}}{z\st{s2}+z\st{s1}}
$。}
\begin{equation}
	\label{eq:totalref_rptp}
	\begin{cases}
		r_p & = \dfrac{p\st{ra}}{p\st{ia}} = \dfrac{m\cos \theta \st{i} +
	\mj n \sinh \psi }{m\cos \theta\st{i} - \mj n \sinh \psi} = 
	\me^{2\mj \phi}\\
	t_p&=\dfrac{p\st{ta}}{p\st{ia}} = \dfrac{2m\cos \theta \st{i}}{m\cos 
		\theta\st{i} - \mj n \sinh \psi} = 2\cos\phi \me^{\mj \phi}
	\end{cases}
	\qc \qty(m = \frac{\rho_2}{\rho_1})
\end{equation}
式中的相位角$\phi$由下列公式给出
\begin{equation}
	\tan \phi = \frac{n\sinh \psi}{m \cos \theta\st{i}} = 
	\frac{\rho_1c_1}{\rho_2c_2} \frac{\sinh \psi}{\cos \theta\st{i}}
	\qc \qty(0<\phi<\frac{\uppi}{2})
\end{equation}
从式(\ref{eq:totalref_rptp})可见,当入射角$\theta\st{i}$大于临界角
$\theta\st{ic}$时,反射波的振幅等于入射的($|r_p|=1$),但入射和反射
声压间存在$2\phi$的相位差。把$r_p=\exp(2\mj \phi)$代入入射和反射声压表达
式(\ref{eq:totalref_pressure}),两者相加再简化,得到入射侧的总声压场
$p_1$和速度场$\vb{v}_1=(v_{1,x}, v_{2,y})$的表达式
\begin{equation}
	\begin{cases}
		p_1&= p\st{i}+p\st{r} = 2p\st{ia}\cos(k_1x\cos\theta\st{i}+\phi)
		\me^{-\mj (k_1y\sin\theta\st{i}-\phi)}\\
		v_{1,x}&= v_{\mi x} + v_{\mathrm{r}x} = 2v\st{ia}\sin
		(k_1x\cos\theta\st{i}+\phi)\cos\theta\st{i}
		\me^{-\mj \qty(k_1y\sin\theta\st{i}-\phi+ \uppi/2)}\\
		v_{1,y}&= v_{\mi y} + v_{\mathrm{r}y} = 2v\st{ia}\sin
		(k_1x\cos\theta\st{i}+\phi)\sin\theta\st{i}
		\me^{-\mj \qty(k_1y\sin\theta\st{i}-\phi)}
	\end{cases}
	\qc \qty(v\st{ia} = \frac{p\st{ia}}{\rho_1c_1})
\end{equation}
式中,$v\st{ia}$是入射声波质点速度振幅。可见,入射侧声场因全反射而在法向
(负$x$方向)形成了驻波,但沿界面($y$方向)仍是传播的。沿界面的传播的相速度和波长分别为
% \begin{equation}
	% \begin{cases}
		% c_{1y} & = \frac{}
	% \end{cases}
% \end{equation}
\end{document}
