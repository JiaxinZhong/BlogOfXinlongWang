\documentclass[UTF8]{ctexbook}

% SCI book 的模板 sty,适用于中文
\usepackage{scibookchn}
\usepackage{scigeneral}

\title{王新龙声学博文}

% start each section on a new page
\let\stdsection\section
\renewcommand\section{\newpage\stdsection}


\begin{document}

\maketitle

\tableofcontents

\chapter{声波的基本性质}

\section{声学量的复数运算\\
Complex Operations for Acoustical Quantities}

\emph{摘要}\ 振动与声学量虽然是可观测的实数物理量,但也可用复数表示。
复数表示不但可极大地简化数学演算,因此无论在线性还是非线性的声学理论中均
广泛采用复数表示声学量。本文简述声学量的复数表示及运算规则,并举例说明。

设小写的$x$为实数量,而大写的$X$为对应的复数量,即$x=\Re(X)$,其中
$\Re(X)$表示取复数量$X$的实部,即
$$
x = \Re(X) = \frac12(X+X^*)
$$
其中右上标的星号「*」表示复共轭操作。显然,实数量$x$与其对应的复数量$X$并非
一一对应。复数量$X$加上任意的线虚数$\mj y$并不影响其意义:
$$
x = \Re(X) = \Re(X+\mj y)
$$

\subsection{加法规则}
若$x$、$y$和$z$为实数量,$a$和$b$是任意的实数,$z=ax+by$。设$x$、$y$和$z$
对应的复数量分别为$X$、$Y$和$Z$,即$x=\Re(X)$、$y=\Re(Y)$和$z=\Re(Z)$,则
因
$$
z = ax+by = \Re(aX+bY) = \Re(Z)
$$
所以,复数量$X$、$Y$和$Z$之间存在以下关系
$$Z=aX+bY$$
现设$x$是时间$t$的函数:$x=x(t)$。它对应的复数量$X$也是时间的函数:$X=
X(t)$。由于导数和积分本质上是线性加法去处,因此存在关系:
$$\dv{t} \Re(X) =\Re\qty(\dv{X}{t})\qc \int \Re(X) \dd{t} = \Re 
\qty(\int X\dd{t})$$

即复数$\dv*{X}{t}$是实数量导数$\dv*{x}{t}$对应的复数量,复数积分$\int X
\dd{t}$是实数积分$\int x\dd{t}$对应的复数量。

\subsection{乘法规则}
现设$z$是$x$和$y$之积,$z=xy$,对应的复数量为$Z$。根据定义,
$$z=xy =\Re(X)\Re(Y) = \Re(\Re(X)Y)=\Re(X\Re(Y))$$
所以,
\begin{subequations}
	\label{eq:complexop_Z}
	\begin{equation}Z= \Re(X)Y = \frac12 (XY+X^*Y)\end{equation}
	\begin{equation}Z=X\Re(Y) = \frac12 (XY+XY^*)\end{equation}
\end{subequations}
再次申明,式(\ref{eq:complexop_Z})中两个表式所给出的$Z$在数学上严格而言是
不等的,只因$Z$的实部才有物理意义,故两种表示在物理上等价。

\subsection{时间简谐量的周期平均值}
复数表示对于简谐声学量的计算尤其有效。假设与时间$t$的依赖关系为$\exp(\mj 
\omega t)$,其中$\omega = 2\uppi /T$为频率,$T$是振动周期,$\mj$是虚数单位
。例如,$X$和$Y$是时间简谐的复量
$$X=X\st{a} \me^{\mj \omega t }\qc Y = Y\st{a}\me^{\mj \omega t}$$
其中,$X\st{a}$和$Y\st{a}$分别为$X$和$Y$的复振幅,是与时间$t$无关的常数。
则有
$$\dd{X}{t}=\mj \omega X\qc \int X\dd{t} = \frac{X}{\mj \omega}$$
根据(\ref{eq:complexop_Z}),此时乘积$z=xy$的复数表达式为
$$Z=\frac12 X\st{a}^* Y\st{a}+\frac12 X\st{a}Y\st{a}\me^{2\mj \omega t}$$
上式第二项有时间依赖关系$\exp(2\mj \omega  t)$,描述了乘积量$z$的瞬态运动。
这是一个二次谐波(频率$2\omega$),其周期平均为零。第一项是与时间无关的
「直流」项,等于复量$Z$的周期平均值:
$$\bar{Z} = \frac1T \int_0^T Z(t) \dd{t} = \frac12 X\st{a}^* Y\st{a} =
\frac12 X^* Y$$
所以,乘积量$z=xy$的周期均值为
$$\bar{z}=\frac1T \int_0^T z\dd{t} = \overline{\Re(Z)} = \Re(\bar{Z})
=\frac12\Re(X^*Y)$$
即,存在以下重要的乘积平均法则
\begin{equation}
	\overline{x(t)y(t)} = \frac12 \Re(X^*Y) \label{eq:complexop_xy}
\end{equation}
对于简谐振动而言,振动物理量本身的时间均值往往为零。但如振动量的乘积等非
线性运算,多产生类似的「直流」项,因而平均值非零。\emph{「直流」和高次谐波
的产生为典型的非线性效应。}

以下以若干典型振动与声学量为例,说明复数运算的应用。

\subsubsection{单振子周期平均能量}
设有质量为$M\st{m}$、弹性系数为$K\st{m}$的单振子,其质点的位移为$x$,速度
$v$,对应的复位移为$X$,复速度为$V$。单振子的势能和动能分别为
$$E\st{p} = \frac12 K\st{m} x^2 = \frac12 K\st{m}\qty[\Re(X)]^2$$
$$E\st{k} = \frac12 M\st{m} v^2 = \frac12 M\st{m} \qty[\Re(V)]^2$$
若单振子仅作简谐振动,则根据式(\ref{eq:complexop_xy}),势能和动能的一个周期
平均值分别为
$$\overline{E\st{p}} = \frac12 K\st{m} \overline{x^2} = \frac14 K\st{m}
|X|^2$$
$$\overline{E\st{k}} = \frac12 M\st{m} \overline{v^2} = \frac14 M\st{m}
|V|^2$$
对于自由振动,
$$V = \mj \omega_0 X\qc \qty(\omega_0 = \sqrt{\frac{K\st{m}}{M\st{m}}})$$
代入前式,知平均势能等于平均动能:
$$\overline{E\st{p}} = \overline{E\st{k}}$$

\subsubsection{声能通量密度和声强的运算}
声能通量密度$\vb{I}$为声压$p$与速度适量$\vb{v}$的乘积:$\vb{I}=p \vb{v}$。
声压、速度和声能通量密度等物理量本身是实量,但均可以用复数表示。在振动
与声的问题中,极大部分情形下所涉及的是复数量及其运算,因此在下文中我们放弃
前面用大写字母表示复量的做法,而一概约定:除非特别说明,\emph{所有的变量符号
均表示对应实量的复量},如声压、复速度和复能量通量密度仍用$p$、$\vb{v}$和
$\vb{I}$表示;如果确实需要用到实数量,则只要对这些复数量取实部操作$\Re$即可
。例如,根据公式(\ref{eq:complexop_Z}),复声能通量密度的公式可写成
\begin{equation}
	\label{eq:complexop_I}
	\vb{I} = \frac12 p\vb{v} + \frac12 p^* \vb{v}
\end{equation}
取实部$\Re(\vb{I})$即为实声能通量密度。

对频率为$\omega$的简谐声波,公式(\ref{eq:complexop_I})的第一项描述声能通量
密度之瞬变,而第二项是「直流」分量,为复声能通量密度之时间均值:
\begin{equation}
	\label{eq:complexop_barI}
	\overline{\vb{I}} = \frac1T \int_t^{t+T}\dd{t} = \frac12 p^* \vb{v}
\end{equation}
取其实部即得到\emph{声强}:
$$\Re(\overline{\vb{I}}) = \frac12 \Re\qty(p^* \vb{v})$$
可见,采用复运算,易分离场量(此处是声能通量密度)的时变和时不变成份。

对于平面行波,传播方向为$\vb{n}$(单位矢量),则媒质质点的振速$\vb{v}
=(p/z_0)\vb{n}$,代入公式(\ref{eq:complexop_I})和(\ref{eq:complexop_barI})
得到
$$\vb{I} = \frac{1}{2z_0}\qty(p^2+|p|^2)\vb{n}$$
$$\overline{\vb{I}} = \frac{1}{2z_0} \overline{\qty(p^2+|p|^2)}
=\frac{|p|^2}{2\rho_0 c_0 }\vb{n}$$
式中$z_0=\rho_0c_0$是媒质的声特性阻抗率。

\subsubsection{声能密度复数运算}
对平面波而言\footnote{编著添加。},用实声学量表示的流体声场的(瞬时)声能
密度为
$$\eps = \frac12 \rho_0 \qty[\vb{v} \vdot \vb{v} +\qty(\frac{p}{z_0})]$$
其中的声压$p$和速度$\vb{v}$是实数量。若全改用复量表示,且仍采用相同的变量
符号,则上式应改为
\begin{align}
	\label{eq:complexop_eps}
	\eps & = \frac14 \rho_0 \qty[(\vb{v}\vdot\vb{v} + \vb{v}^*\vdot
	\vb{v}) +\frac{1}{z_0^2} \qty(p^2+p^* p)]\nonumber\\
	& = \frac14 \rho_0 \qty[\vb{v}\vdot\vb{v}+ \qty(\frac{p}{z_0} )^2]
	+\frac14 \rho_0 \qty(\vb{v}^*\vdot \vb{v} + \qty|\frac{p}{z_0}|^2)
\end{align}
对频率$\omega$的简谐声波,公式(\ref{eq:complexop_eps})后一等式中的第一项描述
声能密度的瞬时变化,而第二项是「直流」分量,为复声能密度之时间周期平均值,
取其实部即为平均声能密度。因此,简谐声场的平均声能密度为
\begin{equation}
	\label{eq:complexop_bareps}
	\overline{\eps} = \frac14 \rho_0 \qty(|\vb{v}|^2 +\frac{|p|^2}{z_0^2})
\end{equation}

对于沿方向$\vb{n}$传播的平面行波,把$\vb{v}=(p/z_0)\vb{n}$代入公式
(\ref{eq:complexop_eps})和(\ref{eq:complexop_bareps})得到
\begin{align*}
	\eps &= \frac{1}{2\rho_0c_0^2} (p^2+|p|^2)\\
	\overline{\eps} &= \frac{1}{2\rho_0c_0^2} |p|^2
\end{align*}


\section{全反射状态的声场与声能流}
英文标题:
Sound Fields and Energy Flux Under Total Reflection

Web:
http://xlwangnu.blog.163.com/blog/static/190719270201192502043374/

\emph{摘要}
全反射是常见声光现象。在媒质交界面上,若入射媒质的声速不如透射媒质的(或折
射率小于1),则当入射角足够大就会发生全反射,入射声能悉数反射。虽然如此,
全反射下声场如何分布?透射媒质中是否存在声场?如存在,取何种波动方式?此
正本文试图回答的问题。分析表明,在全反射下,(1)入射媒质中的总声场在界面法
向呈驻波形式,而在界面切向呈行波形式;(2)仍存在透射声波,但以沿界面传播的
表面波形式存在,但垂直于界面的方向上指数衰减。

若透射媒质的声速$c_2$大于入射媒质的声速$c_1$,且入射角(incident angle)
$\theta\st{i}$大于全反射临界角$\theta\st{ic}$,
\begin{equation}
\theta\st{ic} = \arcsin \frac{c_1}{c_2} = \arcsin n, \qty(n=\frac{c_1}{c_2}
<1)
\end{equation}
声波全部反射。此即著名的\emph{声全反射现象}(total reflection)。与光
的全反射略有不同,声的全反射一般发生在“硬界面”,如从空气到水,从水到固体。
光的全反射则发生在“软表面”---从光密介质入射到光疏介质,如光纤内的光反射,故
而一般称之为全内反射(Total internal reflection)。或问:在全反射状态,声场
分布如何?声能如何传播?此正本文所欲讨论的。

\subsection{声场分布}
如图xx所示,频率为$\omega$的平面声波从声特性阻抗率为$\rho_1c_1$的媒质中以角
度$\theta\st{i}$入射到声特性阻抗率为$\rho_2c_2$的媒质上。
设入射方向和两媒质边界面的法向构成的平面为$(x,y)$平面,$x$轴沿边界内法向,
$y$轴在边界平面上。根据平面行波论,入射(incident)、反射(reflected)和
透射(transmitted)的声场解各具如下形式:
\begin{equation}
	\label{eq:totalref_pressure}
	\begin{cases}
	p\st{i} = p\st{ia} \me^{-\mj k_1 \vb{n}\st{i} \vdot \vb{r}}
	\qc \vb{v}\st{i} = \dfrac{p\st{i}}{\rho_1c_1} \vb{n}\st{i}\\
	p\st{r} = p\st{ra} \me^{-\mj k_1 \vb{n}\st{r} \vdot \vb{r}}
	\qc \vb{v}\st{r} = \dfrac{p\st{r}}{\rho_1c_1} \vb{n}\st{r}
	\qc (p\st{ra} = r_p p\st{ia}) \\
	p\st{t} = p\st{ta} \me^{-\mj k_2 \vb{n}\st{t} \vdot \vb{t}}
	\qc \vb{v}\st{t} = \dfrac{p\st{t}}{\rho_2c_2} \vb{n}\st{t}
	\qc (p\st{ta} = t_p p\st{ta}) 
	\end{cases}
\end{equation}
式中,$p\st{ia}$、$p\st{ra}$和$p\st{ta}$分别为入射、反射和透射(折射)
波的声压幅度,$r_p$和$t_p$分别为声压反射系数和透射系数,$\theta\st{t}$为
折射角(refracted angle),$k_1$和$k_2$分别为入射和透射媒质的波数,
而$\vb{n}\st{i}$、$\vb{n}\st{r}$和$\vb{n}\st{t}$则依次为入射、反射和
折射波的方向矢量,
\begin{equation}
	\vb{n}\st{i}=(\cos \theta\st{i}, \sin \theta\st{i})\qc 
	\vb{n}\st{r}=(-\cos \theta\st{r}, \sin \theta\st{r})\qc 
	\vb{n}\st{t}=(\cos \theta\st{t}, \sin \theta\st{t})
\end{equation}
在$\vb{n}\st{r}$表达式中,已利用了关系$\vb{theta}\st{r}=\theta\st{i}$。为
简洁起见,本文一根省略时间因子$\exp(\mj \omega t)$。

根据Snell定律(折射律),
\begin{equation}
	\label{eq:totalref_snell}
	\sin\theta\st{t} = \frac{\sin\theta\st{i}}{n}\qc (n<1)
\end{equation}
当$\theta>\theta\st{ic}$时,$\sin\theta\st{t}>1$,$\theta\st{t}$不复为实数
而是复数。是以,作复变换:$\theta\st{t} \rightarrow \psi$,
\begin{equation}
	\label{eq:totalref_theta_trans}
	\theta\st{t} = \frac{\uppi}{2}+\mj \psi\qc 
	\begin{cases}
		\sin \theta\st{t} = \cosh \psi\\
		\cos\theta\st{t} = -\mj \sinh \psi
	\end{cases}
\end{equation}
代入折射公式(\ref{eq:totalref_snell}),则 Snell定律改为
\begin{equation}
	\cosh \psi = \frac{\sin \theta\st{i}}{n}
\end{equation}
由此可知,$\psi$具有实数解。当$\theta\st{i}$从$\theta\st{ic}$增至$\uppi/2$
时,$\psi$从$0$单调增大至$\arccosh(1/n)>0$。此处杂提醒读者,数学上,变换
(\ref{eq:totalref_theta_trans})中的$\psi$可正可负。但考虑到下面公式xx(12)
给出的透射声压解$p\st{t}$中,声波必沿$x$正向衰减,$\psi$只能取正值。引入变
换(\ref{eq:totalref_theta_trans})后,全反射($\theta\st{i}>\theta\st{ic}$)
状态下的声压反射系数$r_p$和透射系数$t_p$可表为\footnote{编者著:利用$x=0$
处的边界条件,即$\begin{cases}1+r_p = t_p\\ \dfrac{1-r_p}{z\st{s1}} =
	\dfrac{t_p}{z\st{s2}}\end{cases}$,其中法向声阻抗率分别是
$z\st{s1}= \rho_1c_1/\cos \theta\st{i}, z\st{s2} = \rho_2c_2
/\cos\theta\st{i}$,得到$r_p = \dfrac{z\st{s2}-z\st{s1}}{z\st{s2}+z\st{s1}}
$。}
\begin{equation}
	\label{eq:totalref_rptp}
	\begin{cases}
		r_p & = \dfrac{p\st{ra}}{p\st{ia}} = \dfrac{m\cos \theta \st{i} +
	\mj n \sinh \psi }{m\cos \theta\st{i} - \mj n \sinh \psi} = 
	\me^{2\mj \phi}\\
	t_p&=\dfrac{p\st{ta}}{p\st{ia}} = \dfrac{2m\cos \theta \st{i}}{m\cos 
		\theta\st{i} - \mj n \sinh \psi} = 2\cos\phi \me^{\mj \phi}
	\end{cases}
	\qc \qty(m = \frac{\rho_2}{\rho_1})
\end{equation}
式中的相位角$\phi$由下列公式给出
\begin{equation}
	\tan \phi = \frac{n\sinh \psi}{m \cos \theta\st{i}} = 
	\frac{\rho_1c_1}{\rho_2c_2} \frac{\sinh \psi}{\cos \theta\st{i}}
	\qc \qty(0<\phi<\frac{\uppi}{2})
\end{equation}
从式(\ref{eq:totalref_rptp})可见,当入射角$\theta\st{i}$大于临界角
$\theta\st{ic}$时,反射波的振幅等于入射的($|r_p|=1$),但入射和反射
声压间存在$2\phi$的相位差。把$r_p=\exp(2\mj \phi)$代入入射和反射声压表达
式(\ref{eq:totalref_pressure}),两者相加再简化,得到入射侧的总声压场
$p_1$和速度场$\vb{v}_1=(v_{1,x}, v_{2,y})$的表达式
\begin{equation}
	\begin{cases}
		p_1&= p\st{i}+p\st{r} = 2p\st{ia}\cos(k_1x\cos\theta\st{i}+\phi)
		\me^{-\mj (k_1y\sin\theta\st{i}-\phi)}\\
		v_{1,x}&= v_{\mi x} + v_{\mathrm{r}x} = 2v\st{ia}\sin
		(k_1x\cos\theta\st{i}+\phi)\cos\theta\st{i}
		\me^{-\mj \qty(k_1y\sin\theta\st{i}-\phi+ \uppi/2)}\\
		v_{1,y}&= v_{\mi y} + v_{\mathrm{r}y} = 2v\st{ia}\cos
		(k_1x\cos\theta\st{i}+\phi)\sin\theta\st{i}
		\me^{-\mj \qty(k_1y\sin\theta\st{i}-\phi)}
	\end{cases}
	\qc \qty(v\st{ia} = \frac{p\st{ia}}{\rho_1c_1})
\end{equation}
式中,$v\st{ia}$是入射声波质点速度振幅。可见,入射侧声场因全反射而在法向
(负$x$方向)形成了驻波,但沿界面($y$方向)仍是传播的。沿界面的传播的相速度和波长分别为
% \begin{equation}
	% \begin{cases}
		% c_{1y} & = \frac{}
	% \end{cases}
% \end{equation}
 
\end{document}
